\documentclass[12pt]{article}
\newcommand{\bi}[1]{\hbox{\boldmath{$#1$}}}
\newcommand{\bm}[1]{\hbox{\underline{$#1$}}}
\textheight 23.5 cm
\textwidth 16 cm
\voffset=-22.5 mm
\hoffset=-15 mm

\include{template}

\pagestyle{empty}

\begin{document}
\thispagestyle{empty}


\setcounter{equation}{0}
\centerline{\sc matematično-fizikalni seminar~\thisyear}
\bigskip
\setcounter{equation}{0}
\centerline{\bf 1. naloga: Natančnost števil pri programiranju Gausovega in Eulerjevega integrala }
\bigskip
\bigskip

\noindent Gaussovo verjetnostno porazdelitev
\[
g(x \mid \mu, \sigma) = \frac{1}{\sqrt{2\pi} \sigma}\,
\exp \left(-\frac{(x-\mu)^2}{2\sigma^2}\right) \nonumber
\]
pogosto srečamo v verjetnostnem računu in statistiki. Običajno nas zanima verjetnost  nad intervalom $(\infty, \xi]$:
$$
P(X<\xi \mid \mu, \sigma ) = \int_{-\infty}^{\xi} g(x \mid \mu, \sigma) dx\,,
$$
ki jo izračunamo z uporabo Gaussovega integrala:
$$
{\rm erf}(z) = \frac{2}{\sqrt{\pi}} \,
\int_0^z {\exp}(-t^2)\,dt.
$$
Če uporabimo novo spremenljivko $t=(x-\mu)/(\sqrt{2}\sigma)$, lahko namreč zapišemo:
$$
P(X<\xi \mid \mu, \sigma )   = \frac{1}{2}\left[1+\mathrm{erf}\left(\frac{x-\mu}{\sqrt{2}\sigma}\right)\right]
$$

\begin{figure}[h]
\begin{center}
\includegraphics[width=0.6\textwidth]{800px-Error_Function.svg.png}
\end{center}
\caption{Funkcija \text{erf}(x) na simetričnem intervalu.\label{slika1}}
\end{figure}

Funkcija $\mathrm{erf}(x)$ prikazana na Sliki~\ref{slika1} je liha in zavzema vrednosti na intervalu $(-1,1)$. 
Pri ročnem delu za računanje vrednosti $\mathrm{erf}(x)$ uporabimo tablice, ki jih  najdemo npr. v matematičnem 
priročniku \textit{Abramowitz in Stegun}.  Za računanje z računalnikom pa si moramo pripraviti podprogram. 
Pregledali bomo naslednje možnosti:

\medskip

1. potenčna vrsta
\[
{\rm erf}(z) = {2\over\sqrt{\pi}} \sum_{n=0}^\infty
(-1)^n {z^{2n+1} \over{n!\, (2n+1)}}, \qquad
{\rm erfc}(z) = 1-{\rm erf}(z);
\]

2. asimptotska vrsta
\[
z \sqrt{\pi}\,\exp (z^2)\,{\rm erfc}(z)
\longrightarrow 1 + \sum_{m=1}^\infty {(2m-1)!!\over (-2z^2)^m};
\]

3. racionalna aproksimacija ($z>0$)
\[
{\rm erf}(z) = 1 - (a\,t+b\,t^2+c\,t^3+d\,t^4+e\,t^5)
\, \exp (-z^2)+\varepsilon (z),
\]
\[
t = {1\over 1+pz},\qquad \varepsilon (z) < 1.5 \cdot 10^{-7},
\]

\halign{#\hskip 5em&\qquad$#\>$&\hfil$#$&.#\hfil%
&\qquad$#\>$&\hfil$#$&.#\hfil&\qquad$#\>$&\hfil$#$&.#\hfil\cr
& p=&&32759 11,  & a=&&25482 9592,  & b=&-&28449 6736, \cr
& c=&1&42141 3741, & d=&-1&45315 2027, & e=&1&06140 5429. \cr}

\medskip

\noindent {\it \bf Naloga~1\/}:
Primerjaj metode po uporabnosti! Po možnosti primerjaj tudi potrebne čase za posamične izračune. Opazuj konvergenco asimptotske vrste!
Napravi primerjalno tablico ${\rm erf}(x)$ za $x=$ 0 do 3 v korakih po 0.2 in ${\rm erfc}(x)$ za $x=$ 3 do 8 v korakih po 
0.5 ter pripravi podprogram za izračun $\mathrm{erf}$ za poljubne vrednosti, kjer kombiniraš različne metode za dosežek najboljše natančnosti.

\medskip

\noindent {\it \bf Dodatna naloga\/}:

Implementiraj numerični izračun števila $\pi$ z različnimi zgodovinskimi približki:

1. Leibnizova vrsta za $\arctan{(1)} = \pi / 4$ (14. stoletje)
\[
    \frac{\pi}{4} = 1 - \frac{1}{3} + \frac{1}{5} - \frac{1}{7} + \frac{1}{9} - ...
\]

2. Srinivasa Ramanujan (1910):
\[
    \frac{1}{\pi} = \frac{2\sqrt{2}}{9801} \sum_{k=0}^{\infty} \frac{(4k)! \, (1103 + 26390k)}{(k!)^4 \, 396^{4k}},
\]

ki izračuna dodatnih osem decimalnih mest števila $\pi$ z vsakim členom v vrsti. Ta vrsta je sedaj osnova za najhitrejše algoritme, ki se trenutno uporabljajo za izračun $\pi$. Ocena samo prvega člena prinese vrednost pravilno do sedmih decimalnih mest:
\[
    \frac{9801}{2206\sqrt{2}} \simeq 3.14159273
\]

Vrsta konvergira tako hitro, da že po treh iteracijah dobimo izračun števila $\pi$ natančnega na 16 signifikantnih mest. Naslednje iteracije dodajajo popravke, ki so manjši od natančnosti zapisa števil z dvojno natančnostjo (`double'). Uporabi eno izmed knjižnic, ki omogoča zapis števil s poljubno natančnostjo (npr. \texttt{Decimal} v Python-u) in poizkusi, koliko decimalnih mest števila $\pi$ lahko izračunaš na tak način!

\end{document}
