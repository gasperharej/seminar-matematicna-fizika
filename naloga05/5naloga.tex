\documentclass[12pt]{article}
\newcommand{\bi}[1]{\hbox{\boldmath{$#1$}}}
\newcommand{\bm}[1]{\hbox{\underline{$#1$}}}
\textheight 23 cm
\textwidth 16 cm
\voffset=-20 mm
\hoffset=-15 mm
\usepackage{times}
\usepackage{amsmath}
\pagestyle{empty}
\include{template}

\begin{document}
\thispagestyle{empty}


\setcounter{equation}{0}
\centerline{\sc matematično-fizikalni seminar~\thisyear}
\bigskip
\setcounter{equation}{0}
\centerline{\bf 5. naloga: Lastne vrednosti simetričnega tenzorja}
\bigskip
\bigskip

Simetrični tenzor $\underline{A}$ ranga $j$ ima lastne vrednosti $\lambda_i$
in njim ustrezne lastne vektorje $\bi{x}_i$, določene z
\[
\underline{A}\,\bi{x}_i = \lambda_i\,\bi{x}_i , \qquad i=1, \ldots, j.
\]
Lastne vrednosti lahko poiščemo iz sekularne enačbe, ki ustreza
matriki
\[
\underline{A} - \lambda\cdot\underline{I},
\]
torej iz enačbe podane z determinanto:
\[
{\rm Det}\,(\underline{A}-\lambda\cdot\underline{I}) = 0.
\]
Izberemo metodo za hitro računanje determinant in določimo ničle
te enačbe z bisekcijo ali kako podobno metodo.

Za velike matrike je taka prevedba na algebrajsko enačbo
nepraktična in tudi sicer neuporabna, ker so koreni enačbe
preobčutljivi na napake koeficientov. V takem primeru
računamo raje neposredno in poskušamo matriko $\underline{A}$ diagonalizirati.
Za simetrične matrike obstaja veliko naprednih metod (Jacobijeva, Householderjeva...),
ki jih je enostavno sprogramirati. Metode bazirajo na matematičnemu teoremu, da
se da vsaka simetrična matrika diagonalizirati s pomočjo ortogonalne matrike Q:
\[
A(\mathrm{diag}) = Q^T \cdot A \cdot Q,
\]
kjer potem stolpci matrike Q predstavljajo tudi ustrezne lastne vektorje. Različne
numerične metode nam ponudijo izračun matrike Q.


Obstajajo tudi zelo preprosti načini za ekstrakcijo posameznih lastnih vrednosti:
Največjo lastno vrednost dobimo lahko tudi s potenčno metodo:
s poljubnim začetnim približkom $\bi{x}^{(0)}$
za lastni vektor vstopimo v iteracijo
\[
\bi{y}^{(i+1)} = \underline{A}\,\bi{x}^{(i)},
\quad \bi{x}^{(i+1)} = \frac{1}{a}\,\bi{y}^{(i+1)},
\]
kjer je $a$ normalizacijska konstanta, s katero renormaliziramo
vektorje $\bi{x}^{(i)}$ na stalno velikost. Ko se začne v postopku vektor
\bi{x} `ponavljati' (se malo spreminja), je $a$ dober približek za lastno vrednost, \bi{x}
pa za ustrezni lastni vektor.  Podobno lahko poiščemo tudi najmanjšo
lastno vrednost, ki je enaka največji lastni vrednosti
$\underline{A}^{-1}$. Že znane lastne vrednosti odstranimo iz matrike $\underline{A}$ z \emph{Hotellingovo redukcijo}. (Ko v matriki ranga 4 določimo največjo in najmanjšo lastno vrednost, nam za drugi dve preostane le še
kvadratna enačba, torej redukcija ni več potrebna...)

\medskip

{\it Naloga\/}: Določi lastne vektorje in lastne vrednosti simetričnih matrik in primerjaj rezultate z vgrajenimi funkcijami. Uporabi iterativno QR metodo in razišči, koliko iteracij je potrebno za določeno natančnost. Kako se spremeni hitrost konvergence, če matriko prej preoblikujemo na tridiagonalno matriko?
\begin{itemize}
	\item \[ \begin{bmatrix}
1000. & 0.1 & 0.01 & 0.001 & 0.0001 & 0.00001 & 0.000001 \\
0.1 & 1000. & 0.1 & 0.01 & 0.001 & 0.0001 & 0.00001 \\
0.01 & 0.1 & 1000. & 0.1 & 0.01 & 0.001 & 0.0001\\
0.001 & 0.01 & 0.1 & 1000. & 0.1 & 0.01 & 0.001\\
0.0001 & 0.001 & 0.01 & 0.1 & 1000 & 0.1 & 0.01\\
0.00001 & 0.0001 & 0.001 & 0.01 & 0.1 & 1000.  & 0.1 \\
0.000001 & 0.00001 & 0.0001 & 0.001 & 0.01 & 0.1 & 1000.
	\end{bmatrix} \]
\item \[ \begin{bmatrix}
10^6 & 0.1 & 0.01 & 0.001 & 0.0001 & 0.00001 & 0.000001 \\
0.1 & 10^6 & 0.1 & 0.01 & 0.001 & 0.0001 & 0.00001 \\
0.01 & 0.1 & 10^6 & 0.1 & 0.01 & 0.001 & 0.0001\\
0.001 & 0.01 & 0.1 & 10^6 & 0.1 & 0.01 & 0.001\\
0.0001 & 0.001 & 0.01 & 0.1 & 10^6 & 0.1 & 0.01\\
0.00001 & 0.0001 & 0.001 & 0.01 & 0.1 & 10^6  & 0.1 \\
0.000001 & 0.00001 & 0.0001 & 0.001 & 0.01 & 0.1 & 10^6
	\end{bmatrix} \]
\end{itemize}

{\it Dodatna naloga\/}: Lastne vrednosti in lastne vektorje gornjih matrik na različne načine: potenčna iterativna
metoda, Jacobijeva metoda, iskanje ničel sekularne enačbe, ... Kako hitro različne metode konvergirajo k pravim vrednostim?

\clearpage

\end{document}
