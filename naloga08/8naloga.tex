\documentclass[slovene,11pt,a4paper]{article}
\usepackage[margin=2cm,bottom=3cm,foot=1.5cm]{geometry}
\setlength{\parindent}{0pt}
\setlength{\parskip}{0.5ex}

\usepackage[pdftex]{graphicx}
\DeclareGraphicsExtensions{.pdf,.png}


\usepackage{amsmath}
\usepackage{amsfonts}
\usepackage{mathrsfs}
\usepackage[usenames]{color}
\usepackage[slovene]{babel}
\usepackage[utf8]{inputenc}
\usepackage{siunitx}
\include{template}

\def\phi{\varphi}
\def\eps{\varepsilon}
\def\theta{\vartheta}

\title{
\sc\large Matematično-fizikalni seminar \thisyear\\
\bigskip
\bf\Large 8.~naloga: Robni problem lastnih vrednosti
}
\author{}
\date{}

\begin{document}
\maketitle
\vspace{-1cm}

Pri robnem problemu lastnih vrednosti poznamo diferencialno enačbo
in nekaj robnih pogojev (običajno vsaj toliko, kolikor je red enačbe)
Za rešitev problema moramo v splošnem v enem zamahu določiti
tako (lastne) funkcije, ki ustrezajo danim robnim pogojem,
kot (lastne) vrednosti, ki skupaj zadoščajo diferencialni enačbi.
Reševanje robnih problemov je zato lahko bistveno bolj zapleteno
kot integracija začetnih problemov.


Numerično bomo reševali stacionarno Schr\"odingerjevo enačbo
\begin{equation*}
-\frac{\hbar^2}{2m}\,\Dd[2]{\psi}{x} + V(x)\psi = E\psi
\end{equation*}
za neskončno potencialno jamo ($V(0 < x < a)=0$ na intervalu $[0,a]$ (ker je lažje)
 ter (dodatno) za končno potencialno jamo na intervalu $[-a/2,a/2]$
($V(|x|\ge a/2)=V_0$), za kateri poznamo analitične rešitve;
glej Strnad, {\sl Fizika II\/}.  Dva značilna pristopa, diferenčna
metoda in strelska metoda, nas bosta pripravila na resnejše probleme,
za katere analitičnih rešitev ne poznamo.

Pri {\sl diferenčni metodi\/} razdelimo interval $[0,a]$ na $N$ točk ($x_i = 0 + a/N\cdot i$) ali interval
$[-a/2,a/2]$ na $N$ točk ($x_i = -a/2 + a/N\cdot i$) in prepišemo drugi
krajevni odvod v drugo diferenco, tako da ima brezdimenzijska enačba obliko
\begin{equation*}
\frac{\psi_{i-1} - 2\psi_i + \psi_{i+1}}{h^2} + E\psi_i = 0
\end{equation*}
oziroma
\begin{equation*}
\psi_{i-1} - 2 \psi_i + \psi_{i+1} = \lambda\psi_i \>,
\end{equation*}
kjer je $\lambda=-Eh^2=k^2h^2$ (na faktor $\tfrac{\hbar^2}{2m}$ lahko tudi pozabimo). Diskretizirati je treba tudi robna
pogoja pri  $x=0$ in $x=a$ (ali $x=-a/2$ in $x=a/2$), ki sta v splošnem (in tudi
pri končni jami) mešanega tipa,
\begin{align*}
c_1 \psi_0 + c_2 \frac{\psi_1 - \psi_{-1}}{2h} =& 0 \>, \\
d_1 \psi_N + d_2 \frac{\psi_{N+1} - \psi_{N-1}}{2h} =& 0 \>,
\end{align*}
medtem ko sta pri neskončni jami preprostejša, $\psi_0=\psi_N=0$.
V primerih potencialnih jam tako dobimo tridiagonalni sistem $N$
oziroma $N-1$ linearnih enačb
\begin{equation*}
A \vec{\psi} = \lambda \vec{\psi}
\end{equation*}
za lastne vektorje $\vec{\psi}$ in lastne vrednosti $\lambda$,
ki ga rešujemo z diagonalizacijo.

\smallskip

Pri {\sl strelski metodi\/} začnemo s ``kosinusnim'' začetnim pogojem
v izhodišču $\psi(0)=1$, $\psi'(0)=0$ ali ``sinusnim'' pogojem
$\psi(0)=0$, $\psi'(0)=1$, nato pa z nekim izbranim $E$ (torej $\lambda$) diferencialno
enačbo s poljubno integracijsko shemo (npr.~RK4) integriramo do roba
$x=a/2$ in tam preverimo, ali je izpolnjen drugi robni pogoj, $\psi(a/2)=0$.
Vrednost $E$ ($\lambda$) spreminjamo tako dolgo, dokler robni pogoj ni izpolnjen do
zahtevane natančnosti, in tako dobimo sode in lihe rešitve enačbe
skupaj z ustreznimi lastnimi vrednostmi energije.

\medskip

{\it Naloga\/}: Določi nekaj najnižjih lastnih funkcij in lastnih
vrednosti za neskončno potencialno jamo z diferenčno metodo in metodo streljanja.
Primerjaj natančnost rezultatov z analitičnimi rešitvami v odvisnosti od
relavantnih parametrov (e.g. število točk $N$ na mreži).

\bigskip

{\it Dodatna naloga\/}:
Problem končne jame je s strelsko metodo le trivialna posplošitev
problema neskončne jame: spremeni se le robni pogoj pri $x=a/2$,
ki ima zaradi zahteve po zveznosti in zvezni odvedljivosti valovne
funkcije zdaj obliko $c_1\psi(a/2) + c_2\psi'(a/2) = 0$.
Enostavneje, streljate lahko od $x=0$ (sredina intervala) do `neskončno',
torej do dovolj velike razdalje, kjer je vrednost valovne funckije enaka nič ($\psi(\infty)=0$).
Kaj ima pri diferenčni metodi večjo vlogo pri napaki:
končna natančnost diference, s katero aproksimiramo drugi odvod,
ali zrnatost intervala (končna razsežnost matrike, ki jo
diagonaliziramo)?

\end{document}
