\documentclass[12pt]{article}
\newcommand{\bi}[1]{\hbox{\boldmath{$#1$}}}
\newcommand{\bm}[1]{\hbox{\underline{$#1$}}}
\textheight 23 cm
\textwidth 16 cm
\voffset=-20 mm
\hoffset=-15 mm
\usepackage{times}
\include{template}

\pagestyle{empty}

\begin{document}
\thispagestyle{empty}



\setcounter{equation}{0}
\centerline{\sc matematično-fizikalni seminar~\thisyear}
\bigskip
\centerline{\bf 2. naloga: Iskanje ničel funkcij in preprosta
  numerična integracija v 1D }
\bigskip
\bigskip

Najpreprostejši postopki v numeričnih metodah so iskanje ničel funkcij v eni dimenziji.  Ne obstaja metoda, ki bi sama poiskala
\emph{vse} ničle funkcije, konvergenca k ničli funkcije je vedno
odvisna od začetnih vrednosti algoritma.

Najpreprostejši postopek bi seveda bil kar `prečesavanje'
funkcije z delitvijo definicijskega območja le-te na dovolj gosto
mrežo točk $x_i$ in iskanjem intervala:
$$ f(x_{i+1})\cdot f(x_i) < 0 $$
kjer je v danem  intervalu $[x_i, x_{i+1}]$ zagotovo liho število
ničel. Ta interval lahko potem zopet razdelimo na drobnejše
intervale in znova prečešemo itd.. dokler ne dosežemo željene
natančnosti. Vendar je ta postopek zelo `drag' z računskega stališča, posebej če je definicijsko območje funkcije zelo
veliko. Poleg tega v splošnem ne vemo, ali so izbrani koraki dovolj
majhni, da objamejo ničle funkcij. Tako takšno metodo kvečjemu kombiniramo z
naprednejšimi, v fiziki pa poleg tega pogosto
lahko vsaj nekoliko uganemo ustrezna območja glede na fizikalni
problem, ki ga rešujemo.

Preprosta, a povsem uporabna metoda je t.i. metoda \emph{bisekcije},
ki deluje na naslednji način:
\begin{enumerate}
\item Izberi interval $[a,b]$, kjer velja $$ f(a)\cdot f(b) < 0 , $$ ta
  vsebuje vsaj eno ničlo.
\item Izberi točko na sredi intervala $x=(a+b)/2$. če je $$
  f(a)\cdot f(x) < 0 $$ postavi $b = x$, drugače postavi $a =
  x$.
\item Ponavljaj prejšnji korak, dokler $|b-a| < \epsilon$, kjer je
  $\epsilon$ želena natančnost.
\end{enumerate}

V vsakem koraku se interval zmanjša za polovico, kar je v splošnem
dovolj hitra konvergenca za običajne potrebe. število
potrebnih korakov lahko tudi izračunamo $ n = \log_2 (
|b-a|/\epsilon) $.


Zgornjo metodo lahko prevedemo tudi na \emph{sekantno} metodo in
metodo \emph{regula falsi}, če
predpostavimo, da je funkcija znotraj intervala dovolj pohlevna
(skoraj linearna) in določimo novo točko $x$ namesto z bisekcijo
z linearno interpolacijo: skozi točki $a$ in $b$ potegnemo premico
in pogledamo, kje seka koordinatno os.

Hitrejša metoda je Newton-Rhapsonova metoda, za katero pa moramo
poleg funkcije $f(x)$ poznati tudi analitični izraz za njen odvod
$f'(x)$.
Metoda temelji na Taylorjevem razvoju:
$$f(x+\delta x) = f(x)+f'(x) \delta x + \ldots$$
če predpostavimo, da nam $\delta x$ da želeni premik iz začetne
 točke v ničlo funkcije, in je torej $f(x+\delta x) = 0$
lahko izrazimo $\delta x$:
$$ \delta x = - \frac{f(x)}{f'(x)},$$
kar velja točno le za premice, je pa morda dober približek tudi
za našo funkcijo. Postopek je torej:
\begin{enumerate}
\item Izberi začetno točko $x$.
\item Izračunaj  premik $ \delta x = - \frac{f(x)}{f'(x)}.$
\item Postavi novo točko $x \to x+\delta x$.
\item Ponavljaj zgornja dva koraka dokler $|\delta x |< \epsilon$.
\end{enumerate}

čeprav je konvergenca te metode precej hitrejša od bisekcije, pa
lahko tu zaidemo v težave, če naletimo na točko, kjer gre
odvod funkcije proti ničli; tu nam metoda potem podivja, in gre
interval proti neskončnosti.  Tako je ta metoda pogosto kombinirana
npr. z bisekcijo ki ji poda dodatno robustnost.

Z boljšimi interpolacijskimi približki za določitev nove točke so potem izvedene še
\emph {Riddersova metoda}, \emph{Brentova metoda} ipd. , ki so še ustrezno
hitrejše.

\vskip 1 cm
Drugo področje primerno za uvod v numerične metode v fiziki je
numerična integracija funkcij v eni dimenziji. Tu je interpretacija
preprosta: čim bolje bi radi določili ploščino pod dano
funkcijo $f(x)$.  V ta namen razdelimo interval  integracije $[a,b]$ v
mrežo N točk:
$$x_i = x_1+ h;  ~~~~ i=1,\ldots,N ; ~~~~ x_1 =a , x_{N} = b, $$

pri čemer je korak $h=(b-a)/N$ konstanten. Preproste metode nato
interpolirajo funkcijo $f(x)$ v točkah $x_i$ in $f_i = f(x_i)$ s
polinomi različnih redov (katerih analitične inegrale seveda
poznamo).

Najpreprostejša je \emph{trapezna metoda}, kjer je funkcija $f(x)$
med sosednjima točkama aproksimirana kar s premico (ploskev je
torej trapez). Dobimo:
$$ \int_{x_1}^{x_2} f(x)\, dx = h \left [ \frac{1}{2} f_1 +  \frac{1}{2} f_2 \right ], $$
oziroma za cel interval:
$$ \int_{x_1}^{x_N} f(x)\, dx = h \left [ \frac{1}{2} f_1 + f_2 + f_3
  + \ldots + f_{N-1} + \frac{1}{2} f_N \right ].$$

Ta metoda je seveda točna le za premice, v splošnem je natan\v
cnost $O(h^3 f'')$.

Boljša formula, kjer je interpolacija točna do reda $x^3$, je
\emph{Simpsonova formula}:
$$ \int_{x_1}^{x_3} f(x)\, dx = h \left [ \frac{1}{3} f_1 +
  \frac{4}{3} f_2 +  \frac{1}{3} f_3 \right ], $$

ki seveda potrebuje tri točke in je torej obsega interval dol\v
zine $2h$. Za celoten interval jo potemtakem zlepimo v:
$$ \int_{x_1}^{x_N} f(x)\, dx = h \left [ \frac{1}{3} f_1 +
  \frac{4}{3} f_2 +  \frac{2}{3} f_3 +
  \frac{4}{3} f_4 +  \frac{2}{3} f_5 + \ldots + \frac{2}{3} f_{N-2} +
  \frac{4}{3} f_{N-1} +  \frac{1}{3} f_N  \right ]. $$

Obstajajo še nadaljnje formule za višje rede interpolacij, kot
so \emph{Bode-jeva (Boolova) formula} etc. Za integracijo bolj zahtevnih funkcij
pa so razvite naprednejše metode, kot so kvadraturne formule, kjer
mreža točk ni več konstantna temveč se prilagaja ekstremom
in hitremu spreminjanju funkcij.  Zgornje preproste formule so v
numeričnih izračunih dodatno obdelane za pospešek konvergence
in nato dobimo npr. \emph{Rombergov integracijski algoritem} in
podobno.

\vskip 1cm

\emph{Nalogi}:
\begin{itemize}
\item Z metodami bisekcije in regule falsi poišči ničle nekaj polinomov
  tretjega reda in primerjaj z analitičnimi vrednostmi (uporabi npr. Cardanovo formulo).
  Uporabi tudi kakšen naprednejši algoritem, na primer algoritme v NumPy (\texttt{numpy.roots}) ali
  SciPy \\(\texttt{scipy.optimize.brentq}, Wijngaarden-Dekker-Brent metoda, ena najboljših) in primerjaj hitrost konvergence.
\item S trapezno in Simpsonovo metodo, pa tudi s kakšno 'vgrajeno' v SciPy (ali podobno), določi ploščino
  funkcije $1-x^2$ v intervalu [-1,1] in funkcije $x^3 e^{-x}$ v
  intervalu od nič do neskončno (kaj pomeni tu neskončno - kaj lahko vzameš?).
  Kako dobro slednja konvergira kvrednosti gama funkcije, ki je tu 3!=6 ?
\end{itemize}


\clearpage


\end{document}
