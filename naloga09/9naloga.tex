
\documentclass[slovene,11pt,a4paper]{article}
\usepackage[margin=2cm,bottom=3cm,foot=1.5cm]{geometry}
\setlength{\parindent}{0pt}
\setlength{\parskip}{0.5ex}

\usepackage[pdftex]{graphicx}
\DeclareGraphicsExtensions{.pdf,.png}


\usepackage{amsmath}
\usepackage{amsfonts}
\usepackage{mathrsfs}
\usepackage[usenames]{color}
\usepackage[slovene]{babel}
\usepackage[utf8]{inputenc}
\usepackage{siunitx}
\include{template}

\def\phi{\varphi}
\def\eps{\varepsilon}
\def\theta{\vartheta}


\title{
\sc\large Matematično-fizikalni seminar \thisyear\\
\bigskip
\bf 9. naloga: Začetni problem PDE --- diferenčna metoda}

\author{}
\date{}

\begin{document}
\maketitle
\vspace{-1cm}

Diferenčne metode za aproksimativno reševanje parcialnih
diferencialnih enačb so široko uporabne, saj niso omejene na
preproste geometrije in linearne robne pogoje. Pri aproksimaciji
odvodov s končnimi diferencami se običajno zadovoljimo z najnižjim
možnim redom, formule višjega reda so zelo rade nestabilne.
Tu bomo delovanje diferenčnih metod preskusili na difuzijski in
valovni enačbi.

Temperaturno polje v homogeni neskončni plasti s končno
debelino $a$ (enodimenzio\-nalni problem) je podano z enačbo
\[
\frac{\partial T}{\partial t} = D\,\frac{\partial^2 T}{\partial x^2}
+ \frac{q}{\rho c}\;,\qquad 0<x<a,\qquad D=\frac{\lambda}{\rho c}.
\]
z začetno temperaturno sliko $T(x,t=0)=f(x)$, podani pa morajo biti tudi robni pogoji 
na robovih, $T(x=0,t)=g_0(t)$ ter  $T(x=a,t)=g_1(t)$. Difuzijsko enačbo
najpreprosteje aproksimiramo z (za zapis uporabimo generično funkcijo u(x,t)):
\begin{equation}
\frac{u_{m}^{n+1}-u_{m}^{n}}{\kappa}=D\, \frac{u_{m+1}^{n}-2 u_{m}^{n}+u_{m-1}^{n}}{h^{2}}\, +\, Q \, +O_g\left(\kappa+h^{2}\right)
\end{equation}
kjer štejemo z indeksom $n$, kjer definiramo tudi maksimalni $N$ ter je $0 \leq n \leq N$ za časovne sloje v razmikih $\Delta t = \kappa$,
z indeksom $m$ pa označimo krajevne točke, $0 \leq m \leq M$ v enem
sloju z razmaki $\Delta x = h$. Ob času $t=0$ je z začetnim pogojem podan prvi sloj, $n=0$,
iz njega izračunamo drugi sloj, in tako naprej. Enačba~(1) velja
za vse notranje točke, obe robni točki pa določata robna pogoja.
Ta eksplicitna shema je preprosto izvedljiva 
\[
u_{m}^{n+1}=u_{m}^{n}+r\, \left(u_{m+1}^{n}-2 u_{m}^{n}+u_{m-1}^{n}\right)\, +\, \kappa Q 
\]
in stabilna  za $r=D\kappa/h^2 \leq 1/2$, ni pa posebno točna, saj je v časovnem odvodu le
prvega reda. Nekoliko točnejša postane za $r=1/6$, vendar napredujemo
pri tem koraku zelo počasi, saj predifundira rešitev šele v
$6N^2$ korakih od enega konca krajevnega intervala do drugega.

Boljša je {\sl Crank-Nicholsonova shema}, ki je drugega reda v
času: časovno diferenco na levi strani enačbe~(1) izrazimo z
aritmetično sredino krajevnih diferenc v sloju $n$ in sloju $n+1$.
\[
\frac{u_{m}^{n+1}-u_{m}^{n}}{\kappa}=D\, \frac{1}{2} \left[ \frac{u_{m+1}^{n+1}-2 u_{m}^{n+1}+u_{m-1}^{n+1}}{h^{2}} + \frac{u_{m+1}^{n}-2 u_{m}^{n}+u_{m-1}^{n}}{h^{2}} \right]\, +\, Q \, +O_g\left(\kappa^{2}+h^{2}\right)
\]
Novih vrednosti v sloju $n+1$ ne dobimo eksplicitno, pač pa kot
rešitve tridiagonalnega sistema enačb. Ker je časovna zahtevnost
takega sistema le $\sim M$, se z računom ne zamudimo dosti dlje,
stabilnost pa je zagotovljena za poljubno velik korak. Ker rešujemo
enačbo s konstantnimi koeficienti, je tudi matrika sistema vedno
ista. Preredimo enačbo, da imamo na levi  funkcijo v prihodnosti in na desni v sedanjosti:
\[
u_{m}^{n+1} +\, r \frac{1}{2} \left[ u_{m+1}^{n+1}-2 u_{m}^{n+1}+u_{m-1}^{n+1} \right]= u_{m}^{n} + r\, \frac{1}{2} \left[ u_{m+1}^{n}-2 u_{m}^{n}+u_{m-1}^{n} \right] + \kappa Q . 
\]
To lahko še bolj kompaktno zapišemo v matrični obliki:
Vrednosti  funkcije $u$ v točkah $x_m$ uredimo v (M-1)-dimenzionalen vektor:
\begin{equation*}
\vec{\boldsymbol{u}}^n=(u_1^n,u_2^n,\ldots,u_{M-1}^n)^T
\end{equation*}
in sistem prepišemo v matrično obliko
\begin{equation*}
  \left(\mathsf{I}-\frac{r}{2}\,\mathsf{\underline A}\right)\vec{\boldsymbol{u}}^{n+1}=
  \left(\mathsf{I}+\frac{r}{2}\,\mathsf{\underline A}\right)\vec{\boldsymbol{u}}^{n} + \vec{\boldsymbol{b}} \qquad
  \mathsf{\underline A}=\begin{pmatrix}
  -2 & 1 \\
  1   & -2 & 1 \\
  & 1   & -2 & 1 \\
  & & \ddots & \ddots & \ddots \\
  & & & 1 & -2 & 1 \\
  & & & & 1 & -2
  \end{pmatrix}\>,
\end{equation*}
kjer je vektor $\vec{\boldsymbol{b}}$ napolnjen z robnima pogojema (in po potrebi izviri Q)
\begin{equation*}
  \vec{\boldsymbol{b}}= \left(\frac{r}{2}(u_0^n+u_0^{n+1})+\kappa Q,\kappa Q,\ldots,\frac{r}{2}(u_M^n+u_M^{n+1})+\kappa Q\right)^T
\end{equation*}
Dobili smo torej matrični sistem, ki ga moramo rešiti v vsakem časovnem koraku, da iz stanja $\boldsymbol{u}^i$ dobimo stanje $\boldsymbol{u}^{i+1}$, kar najlažje rešimo kar z Thomasovim algoritmom, ker je matrika tridiagonalna.

\bigskip\noindent
Nihanje strune opisuje enačba
\[
\frac{1}{c^2} \, \frac{\partial^2 u}{\partial t^2}= \frac{\partial^{2} u}{\partial x^{2}}, \quad 0 \leq x \leq a
\]
kjer sta predpisana začetni odmik $u(x,t=0)=f(x)$ in začetna
hitrost $\partial u/\partial t|_{t=0} =g_{0,1}(x)$, na robovih pa je struna vpeta in je torej odmik identično enak nič. 
Za valovno enačbo obstaja
stabilna eksplicitna metoda drugega reda v času. Dobimo jo, če v
enačbi~(1) zapišemo oba odvoda kot
\begin{eqnarray*}
\frac{\partial^2 u}{\partial t^2} & \rightarrow \frac{u_{m}^{n+1}-2 u_{m}^{n}+u_{m}^{n-1}}{\kappa^{2}}+O_g\left(\kappa^{2}\right),\\[3pt]
\frac{\partial^2 u}{\partial x^2} & \rightarrow \frac{u_{m+1}^{n}-2 u_{m}^{n}+u_{m-1}^{n}}{h^{2}}+O_g\left(h^{2}\right).
\end{eqnarray*}
Vrednosti na časovnem sloju $n+1$ izrazimo s slojema $n$ in $n-1$ in analogno za krajevni odvod:
\[u_{m}^{n+1} =  2 u_{m}^{n} - u_{m}^{n-1} +  \left(\frac{c \kappa}{h}\right)^2 \left[u_{m+1}^{n}-2 u_{m}^{n}+u_{m-1}^{n}\right] +
 O_g\left(h^{2}+\kappa^2 \right).
\]
Začetna sloja dobimo iz dveh začetnih pogojev: za funkcijo in za
njen časovni odvod. V posebnem primeru, ko so začetne hitrosti
povsod enake 0, morata biti sloja $n=1$ in $n=-1$ enaka in velja
\[
u_{m}^{1} =
u_{m}^{0} + {\frac{1}{2}}\cdot \left(\frac{c\kappa}{h}\right)^2 \cdot (u_{m+1}^{0} -2 u_{m}^{0}+u_{m-1}^{0}).
\]
Metoda je stabilna za $\kappa c/h \leq 1$ (\emph{Courantov pogoj}), kar pomeni, da zadošča
$\sim N$ korakov za pot vala od enega konca intervala do drugega.
Večji korak omogočajo implicitne metode, vendar z njimi ne
moremo pridobiti dosti.

\medskip

{\it Naloga\/}: Zasledovali bomo časovni razvoj dveh problemov
z robnimi pogoji I. vrste.
\begin{enumerate}
\item Poišči temperaturni profil plasti, ki ima v začetku
povsod temperaturo okolice, le med $0.2a$ in $0.4a$ je segreta na $T_0$.
Nariši temperaturo plasti za različne čase in primerjaj rezultate v
odvisnosti od velikosti koraka $\kappa$ in $h$. Uporabi metodo
FTCS ali Crank-Nicholsonovo metodo. Dodatno uporabi obe metodi in
ju primerjaj.

%\item obliko strune, ki ima v začetku obliko Gaussove funkcije s
%središčem pri $0.4a$ in $\sigma=0.1a$, za čase
%$\omega_1 t/\pi =$ 0 (0.2) 2!
%Začetna hitrost naj bo povsod nič.

\item Poišči obliko strune, ki ima v začetku trikotno obliko
z vrhom pri $x=0.4a$, za čase $\omega_1 t/\pi =$ 0 do 2 za 
različne velikosti koraka $\kappa$ in $h$.
Začetna hitrost naj bo povsod 0.
\end{enumerate}




\end{document}
