\documentclass[12pt]{article}

\setlength{\parindent}{0pt}
\setlength{\parskip}{0.5ex}

\usepackage[pdftex]{graphicx}
\DeclareGraphicsExtensions{.pdf,.png}


\usepackage{amsmath}
\usepackage{amsfonts}
\usepackage{mathrsfs}
\usepackage[usenames]{color}
\usepackage[slovene]{babel}
\usepackage[utf8]{inputenc}
\usepackage{siunitx}

\include{template}


\newcommand{\bi}[1]{\hbox{\boldmath{$#1$}}}
\newcommand{\bm}[1]{\hbox{\underline{$#1$}}}
\textheight 23 cm
\textwidth 16 cm
\voffset=-20 mm
\hoffset=-15 mm
\usepackage{times}

\pagestyle{empty}


\begin{document}
\thispagestyle{empty}


\setcounter{equation}{0}
\centerline{\sc matematično-fizikalni seminar \thisyear}
\bigskip
\setcounter{equation}{0}
\centerline{\bf 6. naloga: Enačbe hoda}
\bigskip
\bigskip

Med najpreprostejše fizikalne modele sodijo enačbe, ki
povezujejo vrednosti spremenljivk sistema z njihovimi časovnimi
spremembami. Tak primer je enačba za časovno odvisnost
temperature majhnega, dobro prevodnega telesa, ko lahko zanemarimo
krajevno odvisnost: enačbo hoda dobimo iz energijskega zakona.
Vzemimo primer žarilne nitke, ki jo grejemo s tokom, pri čemer bomo
uporabili tokovni sunek iz kondenzatorja:
\[
mc \cdot \Dd{T}{t} =
R I_0^2 \cdot {\rm exp}\left(-\frac{2t}{RC}\right) -
\sigma S T^4.
\]
Z vpeljavo brezdimenzijskih spremenljivk dobimo enoparametrično
enačbo
\begin{equation}
\Dd{u}{x} = a\cdot {\rm exp}(-2x) - u^4.
\end{equation}
Zanemarili smo sevalni tok, ki ga nitka prejema iz okolice, zato
se bo temperatura po dovolj dolgem času poljubno približala
absolutni ničli: to pomeni, da rešitve ne kaže vleči predolgo.
Da se izognemo tudi preračunavanju začetne temperature, nas bodo
zanimali samo zelo močni tokovni sunki, ko lahko začetno
notranjo energijo nitke zanemarimo. K enačbi (1) sodi torej
začetni pogoj $u(x=0) = 0$.

Preiskali bomo uporabnost različnih metod za reševanje enačbe
(1). Sledeč fizikalnemu občutku lahko v najbolj grobi inačici
zapišemo (Eulerjeva metoda). To je v bistvu le prepisana aproksimacija za prvi odvod
$y' \approx (y(x+h) - y(x)) / h$, torej
\begin{equation}
y(x+h) = y(x) + h\,\left.\Dd{y}{x}\right|_x \>.
\label{euler}
\end{equation}
Diferencialno enačbo smo prepisali v diferenčno: sistem
spremljamo v ekvidistantnih korakih dolžine $h$. Metoda je
večinoma stabilna, le groba: za večjo natančnost moramo
ustrezno zmanjšati korak.
Za izboljšanje metode zapišimo odvod kot:
$$ \left.\Dd{y}{x}\right|_x = f(x,y) $$ ter $$ x_{n+1} = x_n + h,
~~~ y_n = y(x_n)$$
Heunova metoda (${\cal O}\,(h^3 )$ lokalno) je boljši približek z:
\begin{eqnarray}
\hat{y}_{n+1} & =  & y_n +  h \cdot f(x_n,y_n) \\
y_{n+1} & = & y_n + \frac{h}{2} \left[ f(x_n,y_n) + f(x_{n+1},\hat{y}_{n+1})\right]
\end{eqnarray}
Izvedenka tega je nato Midpoint metoda (tudi ${\cal O}\,(h^3 )$ lokalno):
\begin{eqnarray}
K_1 & = & h \cdot f(x_n,y_n) \\
K_2 & = & h \cdot f(x_n+{1 \over 2}h,y_n+{1 \over 2}K_1) \\
y_{n+1} & = & y_n + K_2
\end{eqnarray}
Le-to lahko potem izboljšamo kot modificirano Midpoint metodo
itd\ldots

V praksi zahtevamo natančnost in numerično učinkovitost,
ki sta neprimerno boljši kot pri opisanih preprostih metodah.
Uporabimo metode, zasnovane na algoritmih prediktor-korektor,
metode višjih redov iz družine Runge-Kutta (z adaptivnimi koraki), ali ekstrapolacijske metode.
Lotimo se jih, če je zahtevana spodobna natančnost (pod 1~\%):
tedaj tudi ne skoparimo, izberemo večinoma četrti red (globalna natančnost je reda ${\cal O}\,(h^4))$.

\medskip
Brez dvoma ena najbolj priljubljenih je metoda RK4,
\begin{align}
k_1 & =
  f\left(x,\,{y}(x)\right) \> {,}\nonumber\\
k_2 & =
  f\left(x+{\textstyle{1\over 2}}h,\,
       {y}(x)+{\textstyle{h\over 2}}k_1\right) \> {,}\nonumber\\
k_3 & =
  f\left(x+{\textstyle{1\over 2}}h,\,
       {y}(x)+{\textstyle{h\over 2}}k_2\right) \> {,}\label{eq:rk4}\\
k_4 & =  f\left(x+h,\,{y}(x)+hk_3\right) \> {,}\nonumber\\
{y}(x+h) & =  {y}(x)
  + {\tfrac{h}{6}}\,\left(k_1 + 2k_2 + 2k_3 + k_4\right) + {\cal O}(h^5)
  \nonumber\> {.}
\end{align}

\bigskip


\noindent
{\it Naloga\/}:
\begin{itemize}
\item za $a = 10$ v enačbi (1) preskusi čim več metod: Eulerjevo, Midpoint, Runge-Kutto 4. reda, Runge-Kutta-Fehlberg  Adams-Bashfort-Moultonov prediktor-korektor ter še katero od boljših metod, ki jih najdete vgrajene v različna programska orodja. Raziščie kakšno natančnost in stabilnost lahko dosežemo s posamično metodo v odvisnosti od širine koraka? Rezultate prikažite s tabelami in/ali grafičnimi prikazi. Izberi metodo (in korak) za izračun družine rešitev pri a=2 do a=18 v korakih (vsaj) po 4! {\it V premislek\/}: kakšno metodo bi uporabil, če bi posebej natančno želel določiti maksimalne temperature in trenutke, ko nastopijo?
\item Z reševanjem navadne diferencialne enačbe $dy/dt=y^2+2t^2$ primerjaj različne metode (e.g. Eulerjeva, Midpoint, klasična Runge-Kutta 4. reda, Runge-Kutta-Fehlberg, Adams-Bashford-Moultonovo, prediktor-korektor).  Privzemi, da je ob t=0, y(0)=1. Primerjajte rešitve za različne širine korakov (e.g. 0.001, 0.01, 0.1). Pozor: rešitev divergira blizu $t = 0.9$.
% tule divergira pri t = 0.9sth !
\end{itemize}


\clearpage


\end{document}
