\documentclass[12pt]{article}
\setlength{\parindent}{0pt}
\setlength{\parskip}{0.5ex}

\usepackage[pdftex]{graphicx}
\DeclareGraphicsExtensions{.pdf,.png}


\usepackage{amsmath}
\usepackage{amsfonts}
\usepackage{mathrsfs}
\usepackage[usenames]{color}
\usepackage[slovene]{babel}
\usepackage[utf8]{inputenc}
\usepackage{siunitx}
\include{template}

\def\phi{\varphi}
\def\eps{\varepsilon}
\def\theta{\vartheta}




\textheight 23 cm
\textwidth 16 cm
\voffset=-20 mm
\hoffset=-15 mm
\usepackage{times}

\pagestyle{empty}

\begin{document}
\thispagestyle{empty}


\setcounter{equation}{0}
\centerline{\sc matematično-fizikalni seminar \thisyear}
\bigskip
\setcounter{equation}{0}

\bigskip
\centerline{\bf 7. naloga: Newtonov zakon}
\bigskip
\bigskip

Gibanje masne točke v polju sil se opiše z diferencialno enačbo
drugega reda
\[
m\, \Dd[2]{x}{t} = F,
\]
če se omejimo na gibanje vzdolž ene same koordinate $x$. Enačba je
seveda enakovredna sistemu enačb prvega reda
\[
m\, \Dd{x}{t} = p \;, \qquad \Dd{p}{t} = F.
\]
Seveda morajo biti na voljo tudi ustrezni začetni pogoji, tipično
$x(t=0)=x_0$ in $dx/dt=v(t=0)=v_0$. Splošnejše gre tu za sistem diferencialnih
enačb drugega reda:
\[
\Dd[n]{y}{x} = f(x,y,y',y'',...),
\]
ki ga lahko prevedemo na sistem enačb prvega reda z uvedbo novih spremenljivk v slogu
gibalne količine pri Netwonovi enačbi ($y'=v,y''=z,...$).

Z nekaj truda se da eksplicitno dokazati, mi pa lahko privzamemo, da so metode za
reševanje enačb hoda (Runge-Kutta 4. reda, prediktor-korektor... ) neposredno uporabne
za reševanje takšnih sistemov enačb in torej aplikabilne v poljubno dimenzijah, kar
naj bi v principu zadovoljilo večino naših zahtev.

Obstaja še posebna kategorija tako imenovanih \emph{simplektičnih} metod, za enačbe, kjer je $f$ le funkcija koordinat, $f(y)$, ki (približno) ohranjajo tudi Hamiltonian,
torej energijo sistema. Najbolj znana metoda je Verlet/St\"ormer/Encke metoda, ki je globalno
natančna do drugega reda in ki točno ohranja tudi vrtilno količino sistema (če je ta v danem problemu smiselna). Rešujemo torej za vsak diskretni korak $n$ velikosti $h$, $x_n=x_0+n \cdot h$:
\[
\Dd[2]{y}{x} = f(y)
\]
in pri diskretizaciji dobimo recept za korak $y_n$ in $v_n=y'_n$:
\begin{eqnarray*}
y_{n+1} &=& y_n + h \cdot v_n + \frac{h^2}{2} \cdot f(y_n) \\
v_{n+1} &=& v_n +  \frac{h}{2} \cdot \left[ f(y_n) + f(y_{n+1}) \right].
\end{eqnarray*}

V drugačnem zapisuje je metoda poznana tudi kot metoda ``Središčne razlike'' (Central Difference Method, CDM), če nas hitrost ne zanima:
\[
y_{n+1} - 2 y_n + y_{n-1} = h^2 \cdot f(y_n),
\]
kjer prvo točko $y_1$ izračunamo po originalni shemi. Metodo CDM lahko uporabljamo tudi
za primere, ko je f tudi funkcija 'časa' x, f(x,y), le da tu simplektičnost ni zagotovljena
(in tudi verjetno ne relevantna).
Za simplektične metode višjih redov je na voljo na primer Forest-Ruth metoda ali Position
Extended Forest-Ruth Like (PEFRL) metoda, ki sta obe globalno četrtega reda in enostavni za
implementacijo.

\medskip

{\it Naloga\/}:
\begin{itemize}
	\item Čim več metod uporabi za izračun preprostega matematičnega nihala $d^2 x/dt^2 = - \sin x$
z začetnim pogojem $x(0)=1$, $v(0)=0$. Primerjaj natančnost v odvisnosti od izbire koraka in poišči korak,
ki zadošča za natančnost na vsaj 3 mesta. Primerjaj tudi periodično stabilnost različnih metod: pusti,
naj teče račun čez 10 ali 20 nihajev in ugotovi, koliko se amplitude nihajev sistematično kvarijo. Izračunaj tudi
Hamiltonian, t.j. energijo $E = 1-\cos x + \frac{1}{2} (dx/dt)^2$ in poglej kako se ohranja s časom.
Dodatno lahko tudi sprogramiraš eliptični integral, ki je analitična rešitev
dane enačbe (seveda pa obstajajo v ustreznih programskih paketih).
	\item \textbf{Dodatno}: Uporabi numerične sheme za račun poševnega meta ob prisotnosti zračnega upora sorazmernega s kvadratom hitrosti. Poišči kot, pri katerem je domet maksimalen!
\end{itemize}


\clearpage


\end{document}
